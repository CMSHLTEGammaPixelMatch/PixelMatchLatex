\subsection{Timing study}

To investigate the effect of the $s$ variable on the timing of the pixel match filter, a module was created which in addition to the normal pixel match window, calculated the value of the $s$ variable.  The timing of the modules was estimated using the log files created by the jobs, and a comparison of the module with and without the $s$ variable is shown in figure \ref{fig:timing}.  The \ttbar sample is used because it had the largest number of jobs and hence largest statistics.  The effect on timing is minimal, so there is effectively no additional overhead due to the addition of the $s$ variable.

\begin{figure}[!bht]
  \begin{center}
    \begin{tabular}{cc}
      \includegraphics[width=0.4\textwidth]{../plots/timing/ttbar_13TeV_25ns_trigger_27_perModuleRun.eps} &
      \includegraphics[width=0.4\textwidth]{../plots/timing/ttbar_13TeV_25ns_trigger_18_7_perModuleRun.eps}
    \end{tabular}
  \caption{The timing of the pixel match filter module with and without the $s$ variable calculation for the single electron trigger (left) and the double electron trigger (right).}
  \label{fig:timing}
  \end{center}
\end{figure}

