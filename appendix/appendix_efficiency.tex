\subsection{Complete list of efficiency plots}

To save space the efficiencies of all remaining variables are presented here.  Note that for barrel electrons in some parts of the spectra, the $\Delta\phi_2$ can give slightly better discrimination than $s_B$.  The efficiencies are shown in figures \ref{fig:eff_phi1_ea}-\ref{fig:eff_rz_ea}.

\begin{figure}[!bht]
  \begin{center}
    \begin{tabular}{cc}
      \includegraphics[width=0.4\textwidth]{../plots/effs/h_multieff_phi1_beam_8_50_trigger_27_B_ea} &
      \includegraphics[width=0.4\textwidth]{../plots/effs/h_multieff_phi1_beam_13_25_trigger_27_B_ea} \\
      \includegraphics[width=0.4\textwidth]{../plots/effs/h_multieff_phi1_beam_8_50_trigger_27_I_ea} &
      \includegraphics[width=0.4\textwidth]{../plots/effs/h_multieff_phi1_beam_13_25_trigger_27_I_ea} \\
      \includegraphics[width=0.4\textwidth]{../plots/effs/h_multieff_phi1_beam_8_50_trigger_27_F_ea} &
      \includegraphics[width=0.4\textwidth]{../plots/effs/h_multieff_phi1_beam_13_25_trigger_27_F_ea} \\
    \end{tabular}
  \caption{Efficiency curves as a function of $\Delta\phi_1$ for barrel electrons (top), intermediate electrons (middle), and forward electrons (bottom) at $8\tev$ (left) and $13\tev$ (right) for events firing the HLT\_Ele27\_WP80\_v13 trigger.}
  \label{fig:eff_phi1_ea}
  \end{center}
\end{figure}
\clearpage
\begin{figure}[!bht]
  \begin{center}
    \begin{tabular}{cc}
      \includegraphics[width=0.4\textwidth]{../plots/effs/h_multieff_phi2_beam_8_50_trigger_27_B_ea} &
      \includegraphics[width=0.4\textwidth]{../plots/effs/h_multieff_phi2_beam_13_25_trigger_27_B_ea} \\
      \includegraphics[width=0.4\textwidth]{../plots/effs/h_multieff_phi2_beam_8_50_trigger_27_I_ea} &
      \includegraphics[width=0.4\textwidth]{../plots/effs/h_multieff_phi2_beam_13_25_trigger_27_I_ea} \\
      \includegraphics[width=0.4\textwidth]{../plots/effs/h_multieff_phi2_beam_8_50_trigger_27_F_ea} &
      \includegraphics[width=0.4\textwidth]{../plots/effs/h_multieff_phi2_beam_13_25_trigger_27_F_ea} \\
    \end{tabular}
  \caption{Efficiency curves as a function of $\Delta\phi_2$ for barrel electrons (top), intermediate electrons (middle), and forward electrons (bottom) at $8\tev$ (left) and $13\tev$ (right) for events firing the HLT\_Ele27\_WP80\_v13 trigger.}
  \label{fig:eff_phi2_ea}
  \end{center}
\end{figure}
\clearpage
\begin{figure}[!bht]
  \begin{center}
    \begin{tabular}{cc}
      \includegraphics[width=0.4\textwidth]{../plots/effs/h_multieff_rz_beam_8_50_trigger_27_B_ea} &
      \includegraphics[width=0.4\textwidth]{../plots/effs/h_multieff_rz_beam_13_25_trigger_27_B_ea} \\
      \includegraphics[width=0.4\textwidth]{../plots/effs/h_multieff_rz_beam_8_50_trigger_27_I_ea} &
      \includegraphics[width=0.4\textwidth]{../plots/effs/h_multieff_rz_beam_13_25_trigger_27_I_ea} \\
      \includegraphics[width=0.4\textwidth]{../plots/effs/h_multieff_rz_beam_8_50_trigger_27_F_ea} &
      \includegraphics[width=0.4\textwidth]{../plots/effs/h_multieff_rz_beam_13_25_trigger_27_F_ea} \\
    \end{tabular}
  \caption{Efficiency curves as a function of $\Delta r/z$ for barrel electrons (top), intermediate electrons (middle), and forward electrons (bottom) at $8\tev$ (left) and $13\tev$ (right) for events firing the HLT\_Ele27\_WP80\_v13 trigger.}
  \label{fig:eff_rz_ea}
  \end{center}
\end{figure}
\clearpage
\clearpage

\subsection{Complete list of efficiency vs fake rate plots and tables}

The efficiency curves for \Zee and \QCD are shown in figures \ref{fig:multieff_phi1_ea}-\ref{fig:multieff_rz_ea}, and it can be seen that the barrel electrons give the best performance, and forward electrons give the worst performance.  The $s$ variables give the best discrimination between the \Zee and \QCD samples.

\begin{figure}[!bht]
  \begin{center}
    \begin{tabular}{cc}
      \includegraphics[width=0.4\textwidth]{../plots/eff_rej/g_multieff_phi1_beam_8_50_trigger_27} &
      \includegraphics[width=0.4\textwidth]{../plots/eff_rej/g_multieff_phi1_beam_8_50_trigger_17_8} \\
      \includegraphics[width=0.4\textwidth]{../plots/eff_rej/g_multieff_phi1_beam_13_25_trigger_27} &
      \includegraphics[width=0.4\textwidth]{../plots/eff_rej/g_multieff_phi1_beam_13_25_trigger_17_8} \\
    \end{tabular}
  \caption{The signal efficiency-fake rate curves as a function of $\Delta\phi_1$ for all electrons at $8\tev$ (top) and $13\tev$ (bottom) for events firing the HLT\_Ele27\_WP80\_v13 trigger (left) and the HLT\_Ele17\_Ele8\_v19 trigger (right).}
  \label{fig:multieff_phi1_ea}
  \end{center}
\end{figure}
\begin{figure}[!bht]
  \begin{center}
    \begin{tabular}{cc}
      \includegraphics[width=0.4\textwidth]{../plots/eff_rej/g_multieff_phi2_beam_8_50_trigger_27} &
      \includegraphics[width=0.4\textwidth]{../plots/eff_rej/g_multieff_phi2_beam_8_50_trigger_17_8} \\
      \includegraphics[width=0.4\textwidth]{../plots/eff_rej/g_multieff_phi2_beam_13_25_trigger_27} &
      \includegraphics[width=0.4\textwidth]{../plots/eff_rej/g_multieff_phi2_beam_13_25_trigger_17_8} \\
    \end{tabular}
  \caption{The signal efficiency-fake rate curves as a function of $\Delta\phi_2$ for all electrons at $8\tev$ (top) and $13\tev$ (bottom) for events firing the HLT\_Ele27\_WP80\_v13 trigger (left) and the HLT\_Ele17\_Ele8\_v19 trigger (right).}
  \label{fig:multieff_phi2_ea}
  \end{center}
\end{figure}
\begin{figure}[!bht]
  \begin{center}
    \begin{tabular}{cc}
      \includegraphics[width=0.4\textwidth]{../plots/eff_rej/g_multieff_rz_beam_8_50_trigger_27} &
      \includegraphics[width=0.4\textwidth]{../plots/eff_rej/g_multieff_rz_beam_8_50_trigger_17_8} \\
      \includegraphics[width=0.4\textwidth]{../plots/eff_rej/g_multieff_rz_beam_13_25_trigger_27} &
      \includegraphics[width=0.4\textwidth]{../plots/eff_rej/g_multieff_rz_beam_13_25_trigger_17_8} \\
    \end{tabular}
  \caption{The signal efficiency-fake rate curves as a function of $\Delta r/z$ for all electrons at $8\tev$ (top) and $13\tev$ (bottom) for events firing the HLT\_Ele27\_WP80\_v13 trigger (left) and the HLT\_Ele17\_Ele8\_v19 trigger (right).}
  \label{fig:multieff_rz_ea}
  \end{center}
\end{figure}
\clearpage

Signal and background efficiencies at different working points are shown in tables \ref{tab:eff_rej_phi1_beam_8_50_sig}-\ref{tab:eff_rej_rz_beam_13_25_bkg}.  In general the $s$ variables give the best performance, although for large signal efficiencies at $13\tev 25\ns$ for barrel electrons, $\Delta\phi_2$ performs marginally better than $s_B$.

\begin{table}[!bht]
  \begin{center}
    \begin{tabular}{c|ccccc|ccccc}
      \hline
      & \multicolumn{5}{c}{HLT\_Ele27\_WP80\_v13} & \multicolumn{5}{c}{HLT\_Ele17\_Ele8\_v19} \\
      \hline
      & \multicolumn{10}{c}{Barrel electrons} \\
      \hline
      $|\Delta\phi_1|\times 10^{3}$  & $<1.6$ & $<9.6$ & $<14.4$ & $<28.8$ & $<35.2$ & $<1.6$ & $<8.8$ & $<14.4$ & $<28.8$ & $<34.4$ \\
      Target $\varepsilon_{sig}$  & $50.0\%$ & $90.0\%$ & $95.0\%$ & $99.0\%$ & $99.5\%$  & $50.0\%$ & $90.0\%$ & $95.0\%$ & $99.0\%$ & $99.5\%$ \\
      Actual $\varepsilon_{sig} $  & $47.9\%$ & $90.3\%$ & $95.1\%$ & $99.0\%$ & $99.5\%$ & $49.5\%$ & $89.8\%$ & $95.2\%$ & $99.0\%$ & $99.5\%$ \\
      Actual $\varepsilon_{bkg}$  & $5.1\%$ & $27.9\%$ & $40.5\%$ & $68.4\%$ & $78.4\%$ & $4.3\%$ & $26.8\%$ & $41.3\%$ & $69.6\%$ & $76.6\%$ \\
      \hline
      & \multicolumn{10}{c}{Intermediate electrons} \\
      \hline
      $|\Delta\phi_1|\times 10^{3}$  & $<3.2$ & $<13.6$ & $<22.4$ & $<59.2$ & $<69.6$ & $<3.2$ & $<15.2$ & $<24.0$ & $<59.2$ & $<68.8$ \\
      Target $\varepsilon_{sig}$  & $50.0\%$ & $90.0\%$ & $95.0\%$ & $99.0\%$ & $99.5\%$  & $50.0\%$ & $90.0\%$ & $95.0\%$ & $99.0\%$ & $99.5\%$ \\
      Actual $\varepsilon_{sig} $  & $50.0\%$ & $89.7\%$ & $95.1\%$ & $99.0\%$ & $99.5\%$ & $47.9\%$ & $89.8\%$ & $95.0\%$ & $99.0\%$ & $99.5\%$ \\
      Actual $\varepsilon_{bkg}$  & $12.6\%$ & $44.3\%$ & $60.3\%$ & $92.6\%$ & $96.4\%$ & $13.5\%$ & $45.9\%$ & $61.9\%$ & $92.2\%$ & $95.5\%$ \\
      \hline
      & \multicolumn{10}{c}{Forward electrons} \\
      \hline
      $|\Delta\phi_1|\times 10^{3}$  & $<2.4$ & $<11.2$ & $<16.8$ & $<48.8$ & $<64.0$ & $<3.2$ & $<12.8$ & $<19.2$ & $<52.0$ & $<66.4$ \\
      Target $\varepsilon_{sig}$  & $50.0\%$ & $90.0\%$ & $95.0\%$ & $99.0\%$ & $99.5\%$  & $50.0\%$ & $90.0\%$ & $95.0\%$ & $99.0\%$ & $99.5\%$ \\
      Actual $\varepsilon_{sig} $  & $47.6\%$ & $90.4\%$ & $95.2\%$ & $99.0\%$ & $99.5\%$ & $54.3\%$ & $90.4\%$ & $94.9\%$ & $99.0\%$ & $99.5\%$ \\
      Actual $\varepsilon_{bkg}$  & $16.8\%$ & $54.0\%$ & $66.5\%$ & $94.5\%$ & $97.4\%$ & $19.5\%$ & $52.5\%$ & $67.1\%$ & $93.5\%$ & $97.2\%$ \\
      \hline
    \end{tabular}
    \caption{Efficiencies for signal and background for targeted signal efficiencies, as a function of $\phi_1$ for $8 \tev 50 \ns.$}
    \label{tab:eff_rej_phi1_beam_8_50_sig}
  \end{center}
\end{table}

\begin{table}[!bht]
  \begin{center}
    \begin{tabular}{c|ccccc|ccccc}
      \hline
      & \multicolumn{5}{c}{HLT\_Ele27\_WP80\_v13} & \multicolumn{5}{c}{HLT\_Ele17\_Ele8\_v19} \\
      \hline
      & \multicolumn{10}{c}{Barrel electrons} \\
      \hline
      $|\Delta\phi_1|\times 10^{3}$  & $<2.4$ & $<10.4$ & $<15.2$ & $<32.0$ & $<36.8$ & $<1.6$ & $<10.4$ & $<15.2$ & $<31.2$ & $<36.8$ \\
      Target $\varepsilon_{sig}$  & $50.0\%$ & $90.0\%$ & $95.0\%$ & $99.0\%$ & $99.5\%$  & $50.0\%$ & $90.0\%$ & $95.0\%$ & $99.0\%$ & $99.5\%$ \\
      Actual $\varepsilon_{sig} $  & $55.1\%$ & $90.0\%$ & $95.0\%$ & $99.0\%$ & $99.5\%$ & $44.5\%$ & $90.3\%$ & $95.0\%$ & $99.0\%$ & $99.4\%$ \\
      Actual $\varepsilon_{bkg}$  & $7.7\%$ & $30.0\%$ & $41.2\%$ & $72.9\%$ & $79.8\%$ & $5.5\%$ & $30.4\%$ & $41.5\%$ & $72.4\%$ & $80.2\%$ \\
      \hline
      & \multicolumn{10}{c}{Intermediate electrons} \\
      \hline
      $|\Delta\phi_1|\times 10^{3}$  & $<3.2$ & $<13.6$ & $<20.0$ & $<55.2$ & $<64.0$ & $<4.0$ & $<15.2$ & $<24.8$ & $<55.2$ & $<64.8$ \\
      Target $\varepsilon_{sig}$  & $50.0\%$ & $90.0\%$ & $95.0\%$ & $99.0\%$ & $99.5\%$  & $50.0\%$ & $90.0\%$ & $95.0\%$ & $99.0\%$ & $99.5\%$ \\
      Actual $\varepsilon_{sig} $  & $47.5\%$ & $90.1\%$ & $95.1\%$ & $99.0\%$ & $99.5\%$ & $52.4\%$ & $89.7\%$ & $95.2\%$ & $99.0\%$ & $99.5\%$ \\
      Actual $\varepsilon_{bkg}$  & $13.4\%$ & $44.9\%$ & $55.7\%$ & $90.2\%$ & $93.6\%$ & $16.9\%$ & $46.3\%$ & $64.1\%$ & $90.9\%$ & $94.7\%$ \\
      \hline
      & \multicolumn{10}{c}{Forward electrons} \\
      \hline
      $|\Delta\phi_1|\times 10^{3}$  & $<3.2$ & $<12.0$ & $<19.2$ & $<48.8$ & $<59.2$ & $<3.2$ & $<13.6$ & $<21.6$ & $<50.4$ & $<60.8$ \\
      Target $\varepsilon_{sig}$  & $50.0\%$ & $90.0\%$ & $95.0\%$ & $99.0\%$ & $99.5\%$  & $50.0\%$ & $90.0\%$ & $95.0\%$ & $99.0\%$ & $99.5\%$ \\
      Actual $\varepsilon_{sig} $  & $51.5\%$ & $89.6\%$ & $94.9\%$ & $99.0\%$ & $99.5\%$ & $49.5\%$ & $89.8\%$ & $94.9\%$ & $99.0\%$ & $99.5\%$ \\
      Actual $\varepsilon_{bkg}$  & $20.8\%$ & $55.2\%$ & $69.2\%$ & $92.1\%$ & $95.5\%$ & $19.1\%$ & $54.5\%$ & $68.6\%$ & $93.8\%$ & $96.2\%$ \\
      \hline
    \end{tabular}
    \caption{Efficiencies for signal and background for targeted signal efficiencies, as a function of $\phi_1$ for $13 \tev 25 \ns.$}
    \label{tab:eff_rej_phi1_beam_13_25_sig}
  \end{center}
\end{table}

\begin{table}[!bht]
  \begin{center}
    \begin{tabular}{c|ccccc|ccccc}
      \hline
      & \multicolumn{5}{c}{HLT\_Ele27\_WP80\_v13} & \multicolumn{5}{c}{HLT\_Ele17\_Ele8\_v19} \\
      \hline
      & \multicolumn{10}{c}{Barrel electrons} \\
      \hline
      $|\Delta\phi_1|\times 10^{3}$  & $<3.2$ & $<10.4$ & $<18.4$ & $<29.6$ & $<53.6$ & $<3.2$ & $<9.6$ & $<18.4$ & $<28.8$ & $<56.8$ \\
      Target $\varepsilon_{bkg}$  & $10.0\%$ & $30.0\%$ & $50.0\%$ & $70.0\%$ & $90.0\%$  & $10.0\%$ & $30.0\%$ & $50.0\%$ & $70.0\%$ & $90.0\%$ \\
      Actual $\varepsilon_{bkg} $  & $9.6\%$ & $30.4\%$ & $49.5\%$ & $69.9\%$ & $90.0\%$ & $11.5\%$ & $29.2\%$ & $50.1\%$ & $69.6\%$ & $89.8\%$ \\
      Actual $\varepsilon_{sig}$  & $67.8\%$ & $91.4\%$ & $97.0\%$ & $99.0\%$ & $99.8\%$ & $69.7\%$ & $90.8\%$ & $97.0\%$ & $99.0\%$ & $99.9\%$ \\
      \hline
      & \multicolumn{10}{c}{Intermediate electrons} \\
      \hline
      $|\Delta\phi_1|\times 10^{3}$  & $<2.4$ & $<8.0$ & $<16.8$ & $<28.8$ & $<52.8$ & $<2.4$ & $<8.0$ & $<16.8$ & $<29.6$ & $<55.2$ \\
      Target $\varepsilon_{bkg}$  & $10.0\%$ & $30.0\%$ & $50.0\%$ & $70.0\%$ & $90.0\%$  & $10.0\%$ & $30.0\%$ & $50.0\%$ & $70.0\%$ & $90.0\%$ \\
      Actual $\varepsilon_{bkg} $  & $9.5\%$ & $29.9\%$ & $50.5\%$ & $70.1\%$ & $89.9\%$ & $10.2\%$ & $29.3\%$ & $49.4\%$ & $69.7\%$ & $90.0\%$ \\
      Actual $\varepsilon_{sig}$  & $41.0\%$ & $78.7\%$ & $92.5\%$ & $96.5\%$ & $98.6\%$ & $39.2\%$ & $76.2\%$ & $91.2\%$ & $96.4\%$ & $98.7\%$ \\
      \hline
      & \multicolumn{10}{c}{Forward electrons} \\
      \hline
      $|\Delta\phi_1|\times 10^{3}$  & $<1.6$ & $<4.8$ & $<9.6$ & $<19.2$ & $<38.4$ & $<1.6$ & $<5.6$ & $<12.0$ & $<20.8$ & $<39.2$ \\
      Target $\varepsilon_{bkg}$  & $10.0\%$ & $30.0\%$ & $50.0\%$ & $70.0\%$ & $90.0\%$  & $10.0\%$ & $30.0\%$ & $50.0\%$ & $70.0\%$ & $90.0\%$ \\
      Actual $\varepsilon_{bkg} $  & $11.9\%$ & $28.4\%$ & $49.8\%$ & $69.6\%$ & $90.3\%$ & $10.2\%$ & $30.6\%$ & $50.3\%$ & $70.0\%$ & $89.8\%$ \\
      Actual $\varepsilon_{sig}$  & $36.1\%$ & $69.9\%$ & $87.5\%$ & $96.0\%$ & $98.6\%$ & $33.9\%$ & $71.8\%$ & $89.3\%$ & $95.5\%$ & $98.5\%$ \\
      \hline
    \end{tabular}
    \caption{Efficiencies for signal and background for targeted background efficiencies, as a function of $\phi_1$ for $8 \tev 50 \ns.$}
    \label{tab:eff_rej_phi1_beam_8_50_bkg}
  \end{center}
\end{table}

\begin{table}[!bht]
  \begin{center}
    \begin{tabular}{c|ccccc|ccccc}
      \hline
      & \multicolumn{5}{c}{HLT\_Ele27\_WP80\_v13} & \multicolumn{5}{c}{HLT\_Ele17\_Ele8\_v19} \\
      \hline
      & \multicolumn{10}{c}{Barrel electrons} \\
      \hline
      $|\Delta\phi_1|\times 10^{3}$  & $<3.2$ & $<10.4$ & $<19.2$ & $<30.4$ & $<53.6$ & $<3.2$ & $<10.4$ & $<19.2$ & $<29.6$ & $<52.8$ \\
      Target $\varepsilon_{bkg}$  & $10.0\%$ & $30.0\%$ & $50.0\%$ & $70.0\%$ & $90.0\%$  & $10.0\%$ & $30.0\%$ & $50.0\%$ & $70.0\%$ & $90.0\%$ \\
      Actual $\varepsilon_{bkg} $  & $10.4\%$ & $30.0\%$ & $49.9\%$ & $70.0\%$ & $90.1\%$ & $11.1\%$ & $30.4\%$ & $50.2\%$ & $69.8\%$ & $89.8\%$ \\
      Actual $\varepsilon_{sig}$  & $63.7\%$ & $90.0\%$ & $96.9\%$ & $98.9\%$ & $99.9\%$ & $65.4\%$ & $90.3\%$ & $96.7\%$ & $98.8\%$ & $99.8\%$ \\
      \hline
      & \multicolumn{10}{c}{Intermediate electrons} \\
      \hline
      $|\Delta\phi_1|\times 10^{3}$  & $<2.4$ & $<8.8$ & $<16.8$ & $<30.4$ & $<55.2$ & $<2.4$ & $<8.8$ & $<16.8$ & $<28.8$ & $<52.0$ \\
      Target $\varepsilon_{bkg}$  & $10.0\%$ & $30.0\%$ & $50.0\%$ & $70.0\%$ & $90.0\%$  & $10.0\%$ & $30.0\%$ & $50.0\%$ & $70.0\%$ & $90.0\%$ \\
      Actual $\varepsilon_{bkg} $  & $10.5\%$ & $29.9\%$ & $50.7\%$ & $70.3\%$ & $90.2\%$ & $10.5\%$ & $31.0\%$ & $49.3\%$ & $70.7\%$ & $89.9\%$ \\
      Actual $\varepsilon_{sig}$  & $38.8\%$ & $79.9\%$ & $93.1\%$ & $97.3\%$ & $99.0\%$ & $37.2\%$ & $77.2\%$ & $91.1\%$ & $96.3\%$ & $98.8\%$ \\
      \hline
      & \multicolumn{10}{c}{Forward electrons} \\
      \hline
      $|\Delta\phi_1|\times 10^{3}$  & $<1.6$ & $<5.6$ & $<11.2$ & $<20.0$ & $<42.4$ & $<1.6$ & $<5.6$ & $<12.0$ & $<22.4$ & $<40.0$ \\
      Target $\varepsilon_{bkg}$  & $10.0\%$ & $30.0\%$ & $50.0\%$ & $70.0\%$ & $90.0\%$  & $10.0\%$ & $30.0\%$ & $50.0\%$ & $70.0\%$ & $90.0\%$ \\
      Actual $\varepsilon_{bkg} $  & $11.1\%$ & $32.3\%$ & $51.2\%$ & $70.2\%$ & $90.1\%$ & $10.4\%$ & $29.5\%$ & $50.6\%$ & $70.0\%$ & $90.0\%$ \\
      Actual $\varepsilon_{sig}$  & $31.9\%$ & $70.5\%$ & $88.2\%$ & $95.2\%$ & $98.7\%$ & $30.0\%$ & $68.3\%$ & $87.9\%$ & $95.3\%$ & $98.5\%$ \\
      \hline
    \end{tabular}
    \caption{Efficiencies for signal and background for targeted background efficiencies, as a function of $\phi_1$ for $13 \tev 25 \ns.$}
    \label{tab:eff_rej_phi1_beam_13_25_bkg}
  \end{center}
\end{table}
\begin{table}[!bht]
  \begin{center}
    \begin{tabular}{c|ccccc|ccccc}
      \hline
      & \multicolumn{5}{c}{HLT\_Ele27\_WP80\_v13} & \multicolumn{5}{c}{HLT\_Ele17\_Ele8\_v19} \\
      \hline
      & \multicolumn{10}{c}{Barrel electrons} \\
      \hline
      $|\Delta\phi_2|\times 10^{3}$  & $<0.1$ & $<0.3$ & $<0.4$ & $<1.6$ & $<2.4$ & $<0.1$ & $<0.2$ & $<0.4$ & $<1.2$ & $<2.1$ \\
      Target $\varepsilon_{sig}$  & $50.0\%$ & $90.0\%$ & $95.0\%$ & $99.0\%$ & $99.5\%$  & $50.0\%$ & $90.0\%$ & $95.0\%$ & $99.0\%$ & $99.5\%$ \\
      Actual $\varepsilon_{sig} $  & $57.3\%$ & $91.1\%$ & $95.0\%$ & $99.0\%$ & $99.5\%$ & $60.3\%$ & $90.3\%$ & $95.0\%$ & $99.0\%$ & $99.5\%$ \\
      Actual $\varepsilon_{bkg}$  & $7.1\%$ & $21.0\%$ & $26.9\%$ & $60.7\%$ & $77.5\%$ & $8.6\%$ & $21.9\%$ & $28.4\%$ & $55.3\%$ & $73.5\%$ \\
      \hline
      & \multicolumn{10}{c}{Intermediate electrons} \\
      \hline
      $|\Delta\phi_2|\times 10^{3}$  & $<0.1$ & $<0.7$ & $<1.4$ & $<3.2$ & $<3.6$ & $<0.2$ & $<0.8$ & $<1.4$ & $<3.2$ & $<3.6$ \\
      Target $\varepsilon_{sig}$  & $50.0\%$ & $90.0\%$ & $95.0\%$ & $99.0\%$ & $99.5\%$  & $50.0\%$ & $90.0\%$ & $95.0\%$ & $99.0\%$ & $99.5\%$ \\
      Actual $\varepsilon_{sig} $  & $46.7\%$ & $90.0\%$ & $94.9\%$ & $99.0\%$ & $99.5\%$ & $53.7\%$ & $89.9\%$ & $94.9\%$ & $98.9\%$ & $99.5\%$ \\
      Actual $\varepsilon_{bkg}$  & $10.2\%$ & $43.2\%$ & $62.5\%$ & $91.4\%$ & $96.4\%$ & $14.7\%$ & $47.3\%$ & $66.3\%$ & $91.1\%$ & $95.7\%$ \\
      \hline
      & \multicolumn{10}{c}{Forward electrons} \\
      \hline
      $|\Delta\phi_2|\times 10^{3}$  & $<0.2$ & $<1.0$ & $<1.8$ & $<3.3$ & $<3.6$ & $<0.2$ & $<1.1$ & $<1.8$ & $<3.3$ & $<3.6$ \\
      Target $\varepsilon_{sig}$  & $50.0\%$ & $90.0\%$ & $95.0\%$ & $99.0\%$ & $99.5\%$  & $50.0\%$ & $90.0\%$ & $95.0\%$ & $99.0\%$ & $99.5\%$ \\
      Actual $\varepsilon_{sig} $  & $48.2\%$ & $90.0\%$ & $95.0\%$ & $99.0\%$ & $99.5\%$ & $53.4\%$ & $90.1\%$ & $95.0\%$ & $99.0\%$ & $99.5\%$ \\
      Actual $\varepsilon_{bkg}$  & $15.5\%$ & $55.4\%$ & $72.8\%$ & $93.4\%$ & $96.7\%$ & $15.8\%$ & $57.8\%$ & $75.2\%$ & $94.2\%$ & $97.2\%$ \\
      \hline
    \end{tabular}
    \caption{Efficiencies for signal and background for targeted signal efficiencies, as a function of $\phi_2$ for $8 \tev 50 \ns.$}
    \label{tab:eff_rej_phi2_beam_8_50_sig}
  \end{center}
\end{table}

\begin{table}[!bht]
  \begin{center}
    \begin{tabular}{c|ccccc|ccccc}
      \hline
      & \multicolumn{5}{c}{HLT\_Ele27\_WP80\_v13} & \multicolumn{5}{c}{HLT\_Ele17\_Ele8\_v19} \\
      \hline
      & \multicolumn{10}{c}{Barrel electrons} \\
      \hline
      $|\Delta\phi_2|\times 10^{3}$  & $<0.1$ & $<0.3$ & $<0.4$ & $<1.6$ & $<2.3$ & $<0.1$ & $<0.2$ & $<0.4$ & $<1.4$ & $<2.2$ \\
      Target $\varepsilon_{sig}$  & $50.0\%$ & $90.0\%$ & $95.0\%$ & $99.0\%$ & $99.5\%$  & $50.0\%$ & $90.0\%$ & $95.0\%$ & $99.0\%$ & $99.5\%$ \\
      Actual $\varepsilon_{sig} $  & $56.2\%$ & $90.5\%$ & $95.1\%$ & $99.0\%$ & $99.5\%$ & $59.5\%$ & $89.4\%$ & $95.3\%$ & $99.0\%$ & $99.5\%$ \\
      Actual $\varepsilon_{bkg}$  & $8.3\%$ & $20.1\%$ & $26.8\%$ & $62.3\%$ & $77.7\%$ & $7.7\%$ & $19.2\%$ & $28.3\%$ & $61.7\%$ & $78.2\%$ \\
      \hline
      & \multicolumn{10}{c}{Intermediate electrons} \\
      \hline
      $|\Delta\phi_2|\times 10^{3}$  & $<0.1$ & $<0.7$ & $<1.4$ & $<3.2$ & $<3.7$ & $<0.1$ & $<0.8$ & $<1.6$ & $<3.4$ & $<3.7$ \\
      Target $\varepsilon_{sig}$  & $50.0\%$ & $90.0\%$ & $95.0\%$ & $99.0\%$ & $99.5\%$  & $50.0\%$ & $90.0\%$ & $95.0\%$ & $99.0\%$ & $99.5\%$ \\
      Actual $\varepsilon_{sig} $  & $46.9\%$ & $90.2\%$ & $95.0\%$ & $99.0\%$ & $99.5\%$ & $46.8\%$ & $90.0\%$ & $95.0\%$ & $99.0\%$ & $99.5\%$ \\
      Actual $\varepsilon_{bkg}$  & $10.1\%$ & $42.6\%$ & $63.2\%$ & $90.6\%$ & $96.0\%$ & $10.5\%$ & $47.0\%$ & $69.1\%$ & $94.1\%$ & $97.5\%$ \\
      \hline
      & \multicolumn{10}{c}{Forward electrons} \\
      \hline
      $|\Delta\phi_2|\times 10^{3}$  & $<0.2$ & $<1.0$ & $<1.9$ & $<3.4$ & $<3.7$ & $<0.2$ & $<1.1$ & $<1.9$ & $<3.4$ & $<3.7$ \\
      Target $\varepsilon_{sig}$  & $50.0\%$ & $90.0\%$ & $95.0\%$ & $99.0\%$ & $99.5\%$  & $50.0\%$ & $90.0\%$ & $95.0\%$ & $99.0\%$ & $99.5\%$ \\
      Actual $\varepsilon_{sig} $  & $49.1\%$ & $90.1\%$ & $95.0\%$ & $99.0\%$ & $99.5\%$ & $48.5\%$ & $90.2\%$ & $94.9\%$ & $99.0\%$ & $99.5\%$ \\
      Actual $\varepsilon_{bkg}$  & $14.8\%$ & $53.3\%$ & $74.1\%$ & $95.5\%$ & $97.4\%$ & $14.1\%$ & $57.6\%$ & $75.0\%$ & $95.3\%$ & $97.6\%$ \\
      \hline
    \end{tabular}
    \caption{Efficiencies for signal and background for targeted signal efficiencies, as a function of $\phi_2$ for $13 \tev 25 \ns.$}
    \label{tab:eff_rej_phi2_beam_13_25_sig}
  \end{center}
\end{table}

\begin{table}[!bht]
  \begin{center}
    \begin{tabular}{c|ccccc|ccccc}
      \hline
      & \multicolumn{5}{c}{HLT\_Ele27\_WP80\_v13} & \multicolumn{5}{c}{HLT\_Ele17\_Ele8\_v19} \\
      \hline
      & \multicolumn{10}{c}{Barrel electrons} \\
      \hline
      $|\Delta\phi_2|\times 10^{3}$  & $<0.1$ & $<0.5$ & $<1.1$ & $<2.0$ & $<3.2$ & $<0.1$ & $<0.4$ & $<1.0$ & $<1.9$ & $<3.2$ \\
      Target $\varepsilon_{bkg}$  & $10.0\%$ & $30.0\%$ & $50.0\%$ & $70.0\%$ & $90.0\%$  & $10.0\%$ & $30.0\%$ & $50.0\%$ & $70.0\%$ & $90.0\%$ \\
      Actual $\varepsilon_{bkg} $  & $10.5\%$ & $30.3\%$ & $50.4\%$ & $70.1\%$ & $90.0\%$ & $8.6\%$ & $29.9\%$ & $49.4\%$ & $70.1\%$ & $90.0\%$ \\
      Actual $\varepsilon_{sig}$  & $71.0\%$ & $96.2\%$ & $98.6\%$ & $99.3\%$ & $99.8\%$ & $60.3\%$ & $95.8\%$ & $98.7\%$ & $99.4\%$ & $99.8\%$ \\
      \hline
      & \multicolumn{10}{c}{Intermediate electrons} \\
      \hline
      $|\Delta\phi_2|\times 10^{3}$  & $<0.1$ & $<0.4$ & $<0.9$ & $<1.8$ & $<3.1$ & $<0.1$ & $<0.4$ & $<0.8$ & $<1.6$ & $<3.1$ \\
      Target $\varepsilon_{bkg}$  & $10.0\%$ & $30.0\%$ & $50.0\%$ & $70.0\%$ & $90.0\%$  & $10.0\%$ & $30.0\%$ & $50.0\%$ & $70.0\%$ & $90.0\%$ \\
      Actual $\varepsilon_{bkg} $  & $10.2\%$ & $30.9\%$ & $49.6\%$ & $69.8\%$ & $90.0\%$ & $11.0\%$ & $29.4\%$ & $50.9\%$ & $70.3\%$ & $89.9\%$ \\
      Actual $\varepsilon_{sig}$  & $46.7\%$ & $82.9\%$ & $92.1\%$ & $96.2\%$ & $98.8\%$ & $45.1\%$ & $79.4\%$ & $91.0\%$ & $95.7\%$ & $98.9\%$ \\
      \hline
      & \multicolumn{10}{c}{Forward electrons} \\
      \hline
      $|\Delta\phi_2|\times 10^{3}$  & $<0.1$ & $<0.4$ & $<0.8$ & $<1.6$ & $<3.0$ & $<0.1$ & $<0.4$ & $<0.8$ & $<1.6$ & $<2.8$ \\
      Target $\varepsilon_{bkg}$  & $10.0\%$ & $30.0\%$ & $50.0\%$ & $70.0\%$ & $90.0\%$  & $10.0\%$ & $30.0\%$ & $50.0\%$ & $70.0\%$ & $90.0\%$ \\
      Actual $\varepsilon_{bkg} $  & $8.4\%$ & $29.7\%$ & $49.2\%$ & $69.7\%$ & $90.1\%$ & $10.5\%$ & $29.8\%$ & $50.5\%$ & $70.0\%$ & $89.7\%$ \\
      Actual $\varepsilon_{sig}$  & $29.7\%$ & $73.3\%$ & $87.7\%$ & $94.1\%$ & $98.4\%$ & $38.4\%$ & $73.8\%$ & $87.3\%$ & $93.4\%$ & $98.0\%$ \\
      \hline
    \end{tabular}
    \caption{Efficiencies for signal and background for targeted background efficiencies, as a function of $\phi_2$ for $8 \tev 50 \ns.$}
    \label{tab:eff_rej_phi2_beam_8_50_bkg}
  \end{center}
\end{table}

\begin{table}[!bht]
  \begin{center}
    \begin{tabular}{c|ccccc|ccccc}
      \hline
      & \multicolumn{5}{c}{HLT\_Ele27\_WP80\_v13} & \multicolumn{5}{c}{HLT\_Ele17\_Ele8\_v19} \\
      \hline
      & \multicolumn{10}{c}{Barrel electrons} \\
      \hline
      $|\Delta\phi_2|\times 10^{3}$  & $<0.1$ & $<0.5$ & $<1.2$ & $<2.0$ & $<3.1$ & $<0.1$ & $<0.4$ & $<1.0$ & $<1.8$ & $<3.1$ \\
      Target $\varepsilon_{bkg}$  & $10.0\%$ & $30.0\%$ & $50.0\%$ & $70.0\%$ & $90.0\%$  & $10.0\%$ & $30.0\%$ & $50.0\%$ & $70.0\%$ & $90.0\%$ \\
      Actual $\varepsilon_{bkg} $  & $11.5\%$ & $29.8\%$ & $50.3\%$ & $70.1\%$ & $90.0\%$ & $11.2\%$ & $30.2\%$ & $50.5\%$ & $70.2\%$ & $90.1\%$ \\
      Actual $\varepsilon_{sig}$  & $69.9\%$ & $96.4\%$ & $98.6\%$ & $99.3\%$ & $99.7\%$ & $73.1\%$ & $95.8\%$ & $98.6\%$ & $99.3\%$ & $99.7\%$ \\
      \hline
      & \multicolumn{10}{c}{Intermediate electrons} \\
      \hline
      $|\Delta\phi_2|\times 10^{3}$  & $<0.1$ & $<0.4$ & $<0.9$ & $<1.8$ & $<3.2$ & $<0.1$ & $<0.4$ & $<0.9$ & $<1.6$ & $<3.0$ \\
      Target $\varepsilon_{bkg}$  & $10.0\%$ & $30.0\%$ & $50.0\%$ & $70.0\%$ & $90.0\%$  & $10.0\%$ & $30.0\%$ & $50.0\%$ & $70.0\%$ & $90.0\%$ \\
      Actual $\varepsilon_{bkg} $  & $10.1\%$ & $29.1\%$ & $49.9\%$ & $69.7\%$ & $90.0\%$ & $10.5\%$ & $29.5\%$ & $49.7\%$ & $69.9\%$ & $90.2\%$ \\
      Actual $\varepsilon_{sig}$  & $46.9\%$ & $80.9\%$ & $91.9\%$ & $95.9\%$ & $98.8\%$ & $46.8\%$ & $79.3\%$ & $90.7\%$ & $95.1\%$ & $98.3\%$ \\
      \hline
      & \multicolumn{10}{c}{Forward electrons} \\
      \hline
      $|\Delta\phi_2|\times 10^{3}$  & $<0.1$ & $<0.4$ & $<0.9$ & $<1.6$ & $<2.8$ & $<0.1$ & $<0.4$ & $<0.8$ & $<1.6$ & $<2.9$ \\
      Target $\varepsilon_{bkg}$  & $10.0\%$ & $30.0\%$ & $50.0\%$ & $70.0\%$ & $90.0\%$  & $10.0\%$ & $30.0\%$ & $50.0\%$ & $70.0\%$ & $90.0\%$ \\
      Actual $\varepsilon_{bkg} $  & $11.3\%$ & $31.0\%$ & $49.5\%$ & $69.6\%$ & $90.1\%$ & $10.9\%$ & $30.3\%$ & $50.5\%$ & $70.1\%$ & $90.1\%$ \\
      Actual $\varepsilon_{sig}$  & $40.8\%$ & $76.2\%$ & $88.1\%$ & $93.8\%$ & $97.6\%$ & $40.3\%$ & $74.4\%$ & $87.0\%$ & $93.6\%$ & $97.7\%$ \\
      \hline
    \end{tabular}
    \caption{Efficiencies for signal and background for targeted background efficiencies, as a function of $\phi_2$ for $13 \tev 25 \ns.$}
    \label{tab:eff_rej_phi2_beam_13_25_bkg}
  \end{center}
\end{table}
\begin{table}[!bht]
  \begin{center}
    \begin{tabular}{c|ccccc|ccccc}
      \hline
      & \multicolumn{5}{c}{HLT\_Ele27\_WP80\_v13} & \multicolumn{5}{c}{HLT\_Ele17\_Ele8\_v19} \\
      \hline
      & \multicolumn{10}{c}{Barrel electrons} \\
      \hline
      $|\Delta z_{B,2}|\times 10^{3}$  & $<2.7$ & $<19.8$ & $<31.5$ & $<61.2$ & $<71.1$ & $<1.8$ & $<15.3$ & $<26.1$ & $<56.7$ & $<68.4$ \\
      Target $\varepsilon_{sig}$  & $50.0\%$ & $90.0\%$ & $95.0\%$ & $99.0\%$ & $99.5\%$  & $50.0\%$ & $90.0\%$ & $95.0\%$ & $99.0\%$ & $99.5\%$ \\
      Actual $\varepsilon_{sig} $  & $53.7\%$ & $90.2\%$ & $95.0\%$ & $99.0\%$ & $99.5\%$ & $46.2\%$ & $90.1\%$ & $94.9\%$ & $99.0\%$ & $99.5\%$ \\
      Actual $\varepsilon_{bkg}$  & $6.5\%$ & $34.3\%$ & $49.2\%$ & $78.4\%$ & $86.7\%$ & $4.6\%$ & $27.0\%$ & $42.1\%$ & $74.3\%$ & $84.8\%$ \\
      \hline
      & \multicolumn{10}{c}{Intermediate electrons} \\
      \hline
      $|\Delta r_{I,2}|\times 10^{3}$  & $<6.0$ & $<28.0$ & $<56.0$ & $<164.0$ & $<180.0$ & $<8.0$ & $<32.0$ & $<62.0$ & $<166.0$ & $<180.0$ \\
      Target $\varepsilon_{sig}$  & $50.0\%$ & $90.0\%$ & $95.0\%$ & $99.0\%$ & $99.5\%$  & $50.0\%$ & $90.0\%$ & $95.0\%$ & $99.0\%$ & $99.5\%$ \\
      Actual $\varepsilon_{sig} $  & $46.0\%$ & $89.6\%$ & $95.1\%$ & $99.0\%$ & $99.5\%$ & $53.5\%$ & $89.8\%$ & $95.0\%$ & $99.0\%$ & $99.5\%$ \\
      Actual $\varepsilon_{bkg}$  & $10.3\%$ & $33.6\%$ & $53.7\%$ & $91.0\%$ & $95.3\%$ & $13.3\%$ & $36.3\%$ & $55.2\%$ & $92.3\%$ & $95.8\%$ \\
      \hline
      & \multicolumn{10}{c}{Forward electrons} \\
      \hline
      $|\Delta r_{F,2}|\times 10^{3}$  & $<6.0$ & $<33.0$ & $<67.5$ & $<127.5$ & $<138.0$ & $<6.0$ & $<36.0$ & $<72.0$ & $<129.0$ & $<139.5$ \\
      Target $\varepsilon_{sig}$  & $50.0\%$ & $90.0\%$ & $95.0\%$ & $99.0\%$ & $99.5\%$  & $50.0\%$ & $90.0\%$ & $95.0\%$ & $99.0\%$ & $99.5\%$ \\
      Actual $\varepsilon_{sig} $  & $51.5\%$ & $90.0\%$ & $94.9\%$ & $99.0\%$ & $99.5\%$ & $49.4\%$ & $90.1\%$ & $95.0\%$ & $99.0\%$ & $99.5\%$ \\
      Actual $\varepsilon_{bkg}$  & $15.1\%$ & $53.9\%$ & $75.0\%$ & $95.0\%$ & $97.3\%$ & $15.5\%$ & $54.9\%$ & $76.0\%$ & $95.2\%$ & $97.5\%$ \\
      \hline
    \end{tabular}
    \caption{Efficiencies for signal and background for targeted signal efficiencies, as a function of $rz_2$ for $8 \tev 50 \ns.$}
    \label{tab:eff_rej_rz_beam_8_50_sig}
  \end{center}
\end{table}

\begin{table}[!bht]
  \begin{center}
    \begin{tabular}{c|ccccc|ccccc}
      \hline
      & \multicolumn{5}{c}{HLT\_Ele27\_WP80\_v13} & \multicolumn{5}{c}{HLT\_Ele17\_Ele8\_v19} \\
      \hline
      & \multicolumn{10}{c}{Barrel electrons} \\
      \hline
      $|\Delta z_{B,2}|\times 10^{3}$  & $<2.7$ & $<20.7$ & $<34.2$ & $<65.7$ & $<73.8$ & $<2.7$ & $<18.0$ & $<31.5$ & $<62.1$ & $<72.0$ \\
      Target $\varepsilon_{sig}$  & $50.0\%$ & $90.0\%$ & $95.0\%$ & $99.0\%$ & $99.5\%$  & $50.0\%$ & $90.0\%$ & $95.0\%$ & $99.0\%$ & $99.5\%$ \\
      Actual $\varepsilon_{sig} $  & $51.9\%$ & $89.9\%$ & $95.0\%$ & $99.0\%$ & $99.5\%$ & $54.9\%$ & $90.2\%$ & $94.9\%$ & $99.0\%$ & $99.5\%$ \\
      Actual $\varepsilon_{bkg}$  & $8.1\%$ & $40.0\%$ & $55.4\%$ & $83.7\%$ & $90.1\%$ & $7.2\%$ & $33.4\%$ & $50.4\%$ & $79.9\%$ & $87.9\%$ \\
      \hline
      & \multicolumn{10}{c}{Intermediate electrons} \\
      \hline
      $|\Delta r_{I,2}|\times 10^{3}$  & $<8.0$ & $<30.0$ & $<48.0$ & $<158.0$ & $<182.0$ & $<8.0$ & $<34.0$ & $<58.0$ & $<166.0$ & $<186.0$ \\
      Target $\varepsilon_{sig}$  & $50.0\%$ & $90.0\%$ & $95.0\%$ & $99.0\%$ & $99.5\%$  & $50.0\%$ & $90.0\%$ & $95.0\%$ & $99.0\%$ & $99.5\%$ \\
      Actual $\varepsilon_{sig} $  & $54.2\%$ & $90.5\%$ & $95.0\%$ & $99.0\%$ & $99.5\%$ & $51.8\%$ & $89.8\%$ & $94.9\%$ & $99.0\%$ & $99.5\%$ \\
      Actual $\varepsilon_{bkg}$  & $13.2\%$ & $35.7\%$ & $48.3\%$ & $90.3\%$ & $96.1\%$ & $12.4\%$ & $39.8\%$ & $57.0\%$ & $92.5\%$ & $97.1\%$ \\
      \hline
      & \multicolumn{10}{c}{Forward electrons} \\
      \hline
      $|\Delta r_{F,2}|\times 10^{3}$  & $<6.0$ & $<34.5$ & $<73.5$ & $<130.5$ & $<139.5$ & $<6.0$ & $<37.5$ & $<72.0$ & $<129.0$ & $<135.0$ \\
      Target $\varepsilon_{sig}$  & $50.0\%$ & $90.0\%$ & $95.0\%$ & $99.0\%$ & $99.5\%$  & $50.0\%$ & $90.0\%$ & $95.0\%$ & $99.0\%$ & $99.5\%$ \\
      Actual $\varepsilon_{sig} $  & $49.8\%$ & $89.8\%$ & $95.1\%$ & $99.0\%$ & $99.5\%$ & $48.6\%$ & $90.0\%$ & $94.9\%$ & $99.0\%$ & $99.5\%$ \\
      Actual $\varepsilon_{bkg}$  & $16.3\%$ & $52.9\%$ & $76.7\%$ & $95.3\%$ & $97.9\%$ & $14.5\%$ & $56.5\%$ & $77.4\%$ & $95.7\%$ & $97.1\%$ \\
      \hline
    \end{tabular}
    \caption{Efficiencies for signal and background for targeted signal efficiencies, as a function of $rz_2$ for $13 \tev 25 \ns.$}
    \label{tab:eff_rej_rz_beam_13_25_sig}
  \end{center}
\end{table}

\begin{table}[!bht]
  \begin{center}
    \begin{tabular}{c|ccccc|ccccc}
      \hline
      & \multicolumn{5}{c}{HLT\_Ele27\_WP80\_v13} & \multicolumn{5}{c}{HLT\_Ele17\_Ele8\_v19} \\
      \hline
      & \multicolumn{10}{c}{Barrel electrons} \\
      \hline
      $|\Delta z_{B,2}|\times 10^{3}$  & $<4.5$ & $<16.2$ & $<32.4$ & $<52.2$ & $<76.5$ & $<4.5$ & $<17.1$ & $<33.3$ & $<51.3$ & $<74.7$ \\
      Target $\varepsilon_{bkg}$  & $10.0\%$ & $30.0\%$ & $50.0\%$ & $70.0\%$ & $90.0\%$  & $10.0\%$ & $30.0\%$ & $50.0\%$ & $70.0\%$ & $90.0\%$ \\
      Actual $\varepsilon_{bkg} $  & $10.7\%$ & $29.6\%$ & $50.4\%$ & $70.4\%$ & $90.3\%$ & $10.6\%$ & $29.8\%$ & $49.8\%$ & $70.1\%$ & $90.0\%$ \\
      Actual $\varepsilon_{sig}$  & $66.0\%$ & $87.9\%$ & $95.3\%$ & $98.4\%$ & $99.7\%$ & $70.2\%$ & $91.2\%$ & $96.7\%$ & $98.7\%$ & $99.7\%$ \\
      \hline
      & \multicolumn{10}{c}{Intermediate electrons} \\
      \hline
      $|\Delta r_{I,2}|\times 10^{3}$  & $<6.0$ & $<24.0$ & $<50.0$ & $<96.0$ & $<160.0$ & $<6.0$ & $<24.0$ & $<52.0$ & $<96.0$ & $<156.0$ \\
      Target $\varepsilon_{bkg}$  & $10.0\%$ & $30.0\%$ & $50.0\%$ & $70.0\%$ & $90.0\%$  & $10.0\%$ & $30.0\%$ & $50.0\%$ & $70.0\%$ & $90.0\%$ \\
      Actual $\varepsilon_{bkg} $  & $10.3\%$ & $30.2\%$ & $50.5\%$ & $70.1\%$ & $90.1\%$ & $10.8\%$ & $29.6\%$ & $50.0\%$ & $70.0\%$ & $90.0\%$ \\
      Actual $\varepsilon_{sig}$  & $46.0\%$ & $87.1\%$ & $94.4\%$ & $96.9\%$ & $98.9\%$ & $44.0\%$ & $85.3\%$ & $94.1\%$ & $96.9\%$ & $98.7\%$ \\
      \hline
      & \multicolumn{10}{c}{Forward electrons} \\
      \hline
      $|\Delta r_{F,2}|\times 10^{3}$  & $<3.0$ & $<13.5$ & $<28.5$ & $<57.0$ & $<109.5$ & $<3.0$ & $<15.0$ & $<30.0$ & $<58.5$ & $<111.0$ \\
      Target $\varepsilon_{bkg}$  & $10.0\%$ & $30.0\%$ & $50.0\%$ & $70.0\%$ & $90.0\%$  & $10.0\%$ & $30.0\%$ & $50.0\%$ & $70.0\%$ & $90.0\%$ \\
      Actual $\varepsilon_{bkg} $  & $8.8\%$ & $29.3\%$ & $49.8\%$ & $69.9\%$ & $89.8\%$ & $8.2\%$ & $30.8\%$ & $49.7\%$ & $70.3\%$ & $89.9\%$ \\
      Actual $\varepsilon_{sig}$  & $30.7\%$ & $76.7\%$ & $88.6\%$ & $93.8\%$ & $98.0\%$ & $29.2\%$ & $77.3\%$ & $88.3\%$ & $93.6\%$ & $97.8\%$ \\
      \hline
    \end{tabular}
    \caption{Efficiencies for signal and background for targeted background efficiencies, as a function of $rz_2$ for $8 \tev 50 \ns.$}
    \label{tab:eff_rej_rz_beam_8_50_bkg}
  \end{center}
\end{table}

\begin{table}[!bht]
  \begin{center}
    \begin{tabular}{c|ccccc|ccccc}
      \hline
      & \multicolumn{5}{c}{HLT\_Ele27\_WP80\_v13} & \multicolumn{5}{c}{HLT\_Ele17\_Ele8\_v19} \\
      \hline
      & \multicolumn{10}{c}{Barrel electrons} \\
      \hline
      $|\Delta z_{B,2}|\times 10^{3}$  & $<3.6$ & $<14.4$ & $<28.8$ & $<48.6$ & $<73.8$ & $<3.6$ & $<15.3$ & $<31.5$ & $<51.3$ & $<74.7$ \\
      Target $\varepsilon_{bkg}$  & $10.0\%$ & $30.0\%$ & $50.0\%$ & $70.0\%$ & $90.0\%$  & $10.0\%$ & $30.0\%$ & $50.0\%$ & $70.0\%$ & $90.0\%$ \\
      Actual $\varepsilon_{bkg} $  & $11.1\%$ & $29.6\%$ & $50.3\%$ & $69.9\%$ & $90.1\%$ & $9.1\%$ & $29.7\%$ & $50.4\%$ & $70.5\%$ & $89.9\%$ \\
      Actual $\varepsilon_{sig}$  & $58.9\%$ & $85.3\%$ & $93.4\%$ & $97.7\%$ & $99.5\%$ & $62.4\%$ & $88.3\%$ & $94.9\%$ & $98.3\%$ & $99.6\%$ \\
      \hline
      & \multicolumn{10}{c}{Intermediate electrons} \\
      \hline
      $|\Delta r_{I,2}|\times 10^{3}$  & $<6.0$ & $<22.0$ & $<50.0$ & $<94.0$ & $<156.0$ & $<6.0$ & $<24.0$ & $<48.0$ & $<86.0$ & $<152.0$ \\
      Target $\varepsilon_{bkg}$  & $10.0\%$ & $30.0\%$ & $50.0\%$ & $70.0\%$ & $90.0\%$  & $10.0\%$ & $30.0\%$ & $50.0\%$ & $70.0\%$ & $90.0\%$ \\
      Actual $\varepsilon_{bkg} $  & $10.3\%$ & $29.3\%$ & $49.9\%$ & $70.1\%$ & $90.2\%$ & $9.8\%$ & $30.3\%$ & $50.5\%$ & $69.8\%$ & $89.7\%$ \\
      Actual $\varepsilon_{sig}$  & $45.2\%$ & $84.8\%$ & $95.3\%$ & $97.3\%$ & $99.0\%$ & $42.8\%$ & $83.6\%$ & $93.6\%$ & $96.6\%$ & $98.6\%$ \\
      \hline
      & \multicolumn{10}{c}{Forward electrons} \\
      \hline
      $|\Delta r_{F,2}|\times 10^{3}$  & $<3.0$ & $<13.5$ & $<31.5$ & $<60.0$ & $<112.5$ & $<4.5$ & $<15.0$ & $<30.0$ & $<58.5$ & $<108.0$ \\
      Target $\varepsilon_{bkg}$  & $10.0\%$ & $30.0\%$ & $50.0\%$ & $70.0\%$ & $90.0\%$  & $10.0\%$ & $30.0\%$ & $50.0\%$ & $70.0\%$ & $90.0\%$ \\
      Actual $\varepsilon_{bkg} $  & $9.6\%$ & $29.6\%$ & $50.1\%$ & $70.2\%$ & $90.1\%$ & $11.5\%$ & $30.1\%$ & $49.4\%$ & $70.3\%$ & $90.2\%$ \\
      Actual $\varepsilon_{sig}$  & $28.9\%$ & $75.4\%$ & $88.9\%$ & $93.6\%$ & $98.0\%$ & $39.5\%$ & $76.6\%$ & $87.6\%$ & $93.2\%$ & $97.6\%$ \\
      \hline
    \end{tabular}
    \caption{Efficiencies for signal and background for targeted background efficiencies, as a function of $rz_2$ for $13 \tev 25 \ns.$}
    \label{tab:eff_rej_rz_beam_13_25_bkg}
  \end{center}
\end{table}
