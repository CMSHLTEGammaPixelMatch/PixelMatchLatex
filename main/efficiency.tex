\subsection{Efficiency as a function of the $s$ variables}

Signal efficiencies and fake rates are estimated relative to the small window by counting events in which at least one electron candidates passes the selections.  Efficiencies as a function of the $s$ variables are shown in figure \ref{fig:eff_s_ea}.  The efficiences for the $t\bar{t}$ sample are shown to be marginally worse for intermediate and forward electrons than those in the $Z\to ee$ sample, indicating slightly worse performance for events with large numbers of jets.  The efficiency curves are similar between $8\tev$ and $13\tev$ samples.

The efficiency vs fake rates curves as a function of $s$ are shown in figure \ref{fig:multieff_s_ea} (with single electron trigger samples used as a crosscheck) and the efficiencies for given working points are shown in tables \ref{tab:eff_rej_s_beam_8_50_sig}-\ref{tab:eff_rej_s_beam_13_25_bkg}.  The chosen working points are $\varepsilon_{sig}=50\%, 90\%, 95\%, 99\%, 99.5\%$, and $\varepsilon_{bkg}=10\%, 30\%, 50\%, 70\%, 90\%$.  Typical fake rates for events firing the HLT\_Ele27\_WP80\_v13 trigger at a working point of $\varepsilon_{sig}=95\%$ are $33\% (35\%)$ at $8\tev(13\tev)$ for barrel electrons, $52\% (42\%)$ at $8\tev(13\tev)$ for intermediate electrons, and $66\% (69\%)$ at $8\tev(13\tev)$ for forward electrons, indicating good separation of signal and backgrounds.  Results are similar between the events firing the single and the events firing double electron triggers.

\input{snippets/effs_plots_ea_s}
\input{snippets/eff_rej_plots_ea_s}
\begin{table}[!bht]
  \begin{center}
    \begin{tabular}{c|ccccc|ccccc}
      \hline
      & \multicolumn{5}{c}{HLT\_Ele27\_WP80\_v13} & \multicolumn{5}{c}{HLT\_Ele17\_Ele8\_v19} \\
      \hline
      & \multicolumn{10}{c}{Barrel electrons} \\
      \hline
      $|\tanh{s_B/10}|$  & $<0.06$ & $<0.2$ & $<0.3$ & $<0.5$ & $<0.6$ & $<0.04$ & $<0.2$ & $<0.3$ & $<0.5$ & $<0.6$ \\
      Target $\varepsilon_{sig}$  & $50.0\%$ & $90.0\%$ & $95.0\%$ & $99.0\%$ & $99.5\%$  & $50.0\%$ & $90.0\%$ & $95.0\%$ & $99.0\%$ & $99.5\%$ \\
      Actual $\varepsilon_{sig} $  & $52.9\%$ & $89.7\%$ & $95.0\%$ & $99.0\%$ & $99.5\%$ & $48.5\%$ & $89.8\%$ & $95.1\%$ & $99.0\%$ & $99.5\%$ \\
      Actual $\varepsilon_{bkg}$  & $1.6\%$ & $18.9\%$ & $32.9\%$ & $63.2\%$ & $77.2\%$ & $1.1\%$ & $14.8\%$ & $28.2\%$ & $59.3\%$ & $73.6\%$ \\
      \hline
      & \multicolumn{10}{c}{Intermediate electrons} \\
      \hline
      $|\tanh{s_I/10}|$  & $<0.07$ & $<0.2$ & $<0.4$ & $<0.7$ & $<0.7$ & $<0.07$ & $<0.2$ & $<0.4$ & $<0.7$ & $<0.7$ \\
      Target $\varepsilon_{sig}$  & $50.0\%$ & $90.0\%$ & $95.0\%$ & $99.0\%$ & $99.5\%$  & $50.0\%$ & $90.0\%$ & $95.0\%$ & $99.0\%$ & $99.5\%$ \\
      Actual $\varepsilon_{sig} $  & $53.5\%$ & $90.0\%$ & $95.1\%$ & $99.0\%$ & $99.5\%$ & $50.4\%$ & $89.7\%$ & $95.0\%$ & $99.0\%$ & $99.5\%$ \\
      Actual $\varepsilon_{bkg}$  & $3.9\%$ & $26.5\%$ & $52.2\%$ & $92.3\%$ & $97.0\%$ & $3.1\%$ & $26.0\%$ & $51.3\%$ & $91.6\%$ & $95.9\%$ \\
      \hline
      & \multicolumn{10}{c}{Forward electrons} \\
      \hline
      $|\tanh{s_F/10}|$  & $<0.06$ & $<0.2$ & $<0.3$ & $<0.6$ & $<0.7$ & $<0.06$ & $<0.2$ & $<0.3$ & $<0.6$ & $<0.7$ \\
      Target $\varepsilon_{sig}$  & $50.0\%$ & $90.0\%$ & $95.0\%$ & $99.0\%$ & $99.5\%$  & $50.0\%$ & $90.0\%$ & $95.0\%$ & $99.0\%$ & $99.5\%$ \\
      Actual $\varepsilon_{sig} $  & $49.5\%$ & $89.5\%$ & $95.1\%$ & $99.0\%$ & $99.5\%$ & $46.7\%$ & $90.3\%$ & $95.0\%$ & $99.0\%$ & $99.5\%$ \\
      Actual $\varepsilon_{bkg}$  & $7.6\%$ & $46.1\%$ & $65.5\%$ & $94.4\%$ & $97.1\%$ & $5.2\%$ & $46.3\%$ & $64.9\%$ & $94.0\%$ & $97.1\%$ \\
      \hline
    \end{tabular}
    \caption{Efficiencies for signal and background for targeted signal efficiencies, as a function of $s$ for $8 \tev 50 \ns.$}
    \label{tab:eff_rej_s_beam_8_50_sig}
  \end{center}
\end{table}

\begin{table}[!bht]
  \begin{center}
    \begin{tabular}{c|ccccc|ccccc}
      \hline
      & \multicolumn{5}{c}{HLT\_Ele27\_WP80\_v13} & \multicolumn{5}{c}{HLT\_Ele17\_Ele8\_v19} \\
      \hline
      & \multicolumn{10}{c}{Barrel electrons} \\
      \hline
      $|\tanh{s_B/10}|$  & $<0.06$ & $<0.2$ & $<0.4$ & $<0.5$ & $<0.6$ & $<0.06$ & $<0.2$ & $<0.3$ & $<0.5$ & $<0.6$ \\
      Target $\varepsilon_{sig}$  & $50.0\%$ & $90.0\%$ & $95.0\%$ & $99.0\%$ & $99.5\%$  & $50.0\%$ & $90.0\%$ & $95.0\%$ & $99.0\%$ & $99.5\%$ \\
      Actual $\varepsilon_{sig} $  & $48.4\%$ & $89.8\%$ & $94.9\%$ & $99.0\%$ & $99.5\%$ & $50.7\%$ & $90.3\%$ & $94.9\%$ & $99.0\%$ & $99.5\%$ \\
      Actual $\varepsilon_{bkg}$  & $2.0\%$ & $18.7\%$ & $35.2\%$ & $67.1\%$ & $79.2\%$ & $1.4\%$ & $19.9\%$ & $32.4\%$ & $67.0\%$ & $78.5\%$ \\
      \hline
      & \multicolumn{10}{c}{Intermediate electrons} \\
      \hline
      $|\tanh{s_I/10}|$  & $<0.07$ & $<0.2$ & $<0.3$ & $<0.6$ & $<0.7$ & $<0.07$ & $<0.2$ & $<0.4$ & $<0.7$ & $<0.7$ \\
      Target $\varepsilon_{sig}$  & $50.0\%$ & $90.0\%$ & $95.0\%$ & $99.0\%$ & $99.5\%$  & $50.0\%$ & $90.0\%$ & $95.0\%$ & $99.0\%$ & $99.5\%$ \\
      Actual $\varepsilon_{sig} $  & $51.2\%$ & $89.9\%$ & $95.0\%$ & $99.0\%$ & $99.5\%$ & $47.3\%$ & $90.1\%$ & $94.9\%$ & $99.0\%$ & $99.5\%$ \\
      Actual $\varepsilon_{bkg}$  & $4.2\%$ & $22.1\%$ & $41.6\%$ & $89.2\%$ & $93.8\%$ & $3.8\%$ & $29.3\%$ & $53.8\%$ & $92.5\%$ & $94.9\%$ \\
      \hline
      & \multicolumn{10}{c}{Forward electrons} \\
      \hline
      $|\tanh{s_F/10}|$  & $<0.07$ & $<0.2$ & $<0.3$ & $<0.6$ & $<0.7$ & $<0.07$ & $<0.2$ & $<0.3$ & $<0.6$ & $<0.7$ \\
      Target $\varepsilon_{sig}$  & $50.0\%$ & $90.0\%$ & $95.0\%$ & $99.0\%$ & $99.5\%$  & $50.0\%$ & $90.0\%$ & $95.0\%$ & $99.0\%$ & $99.5\%$ \\
      Actual $\varepsilon_{sig} $  & $52.5\%$ & $89.7\%$ & $95.1\%$ & $99.0\%$ & $99.5\%$ & $50.5\%$ & $89.9\%$ & $95.2\%$ & $98.9\%$ & $99.5\%$ \\
      Actual $\varepsilon_{bkg}$  & $7.7\%$ & $45.4\%$ & $69.0\%$ & $92.1\%$ & $95.5\%$ & $8.2\%$ & $49.4\%$ & $67.8\%$ & $93.8\%$ & $96.5\%$ \\
      \hline
    \end{tabular}
    \caption{Efficiencies for signal and background for targeted signal efficiencies, as a function of $s$ for $13 \tev 25 \ns.$}
    \label{tab:eff_rej_s_beam_13_25_sig}
  \end{center}
\end{table}

\begin{table}[!bht]
  \begin{center}
    \begin{tabular}{c|ccccc|ccccc}
      \hline
      & \multicolumn{5}{c}{HLT\_Ele27\_WP80\_v13} & \multicolumn{5}{c}{HLT\_Ele17\_Ele8\_v19} \\
      \hline
      & \multicolumn{10}{c}{Barrel electrons} \\
      \hline
      $|\tanh{s_B/10}|$  & $<0.2$ & $<0.3$ & $<0.5$ & $<0.6$ & $<0.7$ & $<0.2$ & $<0.3$ & $<0.5$ & $<0.6$ & $<0.7$ \\
      Target $\varepsilon_{bkg}$  & $10.0\%$ & $30.0\%$ & $50.0\%$ & $70.0\%$ & $90.0\%$  & $10.0\%$ & $30.0\%$ & $50.0\%$ & $70.0\%$ & $90.0\%$ \\
      Actual $\varepsilon_{bkg} $  & $10.3\%$ & $30.4\%$ & $50.3\%$ & $70.5\%$ & $90.1\%$ & $10.2\%$ & $30.0\%$ & $49.6\%$ & $70.6\%$ & $90.4\%$ \\
      Actual $\varepsilon_{sig}$  & $82.9\%$ & $94.2\%$ & $98.0\%$ & $99.3\%$ & $99.9\%$ & $85.3\%$ & $95.8\%$ & $98.3\%$ & $99.4\%$ & $99.9\%$ \\
      \hline
      & \multicolumn{10}{c}{Intermediate electrons} \\
      \hline
      $|\tanh{s_I/10}|$  & $<0.1$ & $<0.2$ & $<0.4$ & $<0.5$ & $<0.6$ & $<0.1$ & $<0.2$ & $<0.4$ & $<0.5$ & $<0.6$ \\
      Target $\varepsilon_{bkg}$  & $10.0\%$ & $30.0\%$ & $50.0\%$ & $70.0\%$ & $90.0\%$  & $10.0\%$ & $30.0\%$ & $50.0\%$ & $70.0\%$ & $90.0\%$ \\
      Actual $\varepsilon_{bkg} $  & $9.6\%$ & $29.8\%$ & $50.5\%$ & $69.5\%$ & $90.2\%$ & $9.2\%$ & $29.0\%$ & $49.3\%$ & $70.0\%$ & $90.3\%$ \\
      Actual $\varepsilon_{sig}$  & $73.6\%$ & $91.2\%$ & $94.9\%$ & $96.7\%$ & $98.7\%$ & $73.4\%$ & $91.0\%$ & $94.8\%$ & $96.8\%$ & $98.8\%$ \\
      \hline
      & \multicolumn{10}{c}{Forward electrons} \\
      \hline
      $|\tanh{s_F/10}|$  & $<0.07$ & $<0.1$ & $<0.2$ & $<0.3$ & $<0.5$ & $<0.08$ & $<0.2$ & $<0.2$ & $<0.3$ & $<0.5$ \\
      Target $\varepsilon_{bkg}$  & $10.0\%$ & $30.0\%$ & $50.0\%$ & $70.0\%$ & $90.0\%$  & $10.0\%$ & $30.0\%$ & $50.0\%$ & $70.0\%$ & $90.0\%$ \\
      Actual $\varepsilon_{bkg} $  & $10.4\%$ & $30.5\%$ & $50.2\%$ & $69.9\%$ & $89.7\%$ & $9.0\%$ & $30.9\%$ & $51.3\%$ & $70.7\%$ & $90.2\%$ \\
      Actual $\varepsilon_{sig}$  & $56.4\%$ & $81.4\%$ & $91.3\%$ & $95.8\%$ & $98.6\%$ & $58.8\%$ & $82.8\%$ & $91.6\%$ & $96.1\%$ & $98.6\%$ \\
      \hline
    \end{tabular}
    \caption{Efficiencies for signal and background for targeted background efficiencies, as a function of $s$ for $8 \tev 50 \ns.$}
    \label{tab:eff_rej_s_beam_8_50_bkg}
  \end{center}
\end{table}

\begin{table}[!bht]
  \begin{center}
    \begin{tabular}{c|ccccc|ccccc}
      \hline
      & \multicolumn{5}{c}{HLT\_Ele27\_WP80\_v13} & \multicolumn{5}{c}{HLT\_Ele17\_Ele8\_v19} \\
      \hline
      & \multicolumn{10}{c}{Barrel electrons} \\
      \hline
      $|\tanh{s_B/10}|$  & $<0.2$ & $<0.3$ & $<0.5$ & $<0.6$ & $<0.7$ & $<0.2$ & $<0.3$ & $<0.4$ & $<0.6$ & $<0.7$ \\
      Target $\varepsilon_{bkg}$  & $10.0\%$ & $30.0\%$ & $50.0\%$ & $70.0\%$ & $90.0\%$  & $10.0\%$ & $30.0\%$ & $50.0\%$ & $70.0\%$ & $90.0\%$ \\
      Actual $\varepsilon_{bkg} $  & $10.4\%$ & $30.2\%$ & $50.4\%$ & $69.1\%$ & $89.9\%$ & $9.8\%$ & $29.5\%$ & $49.5\%$ & $70.6\%$ & $90.3\%$ \\
      Actual $\varepsilon_{sig}$  & $81.2\%$ & $93.9\%$ & $97.5\%$ & $99.1\%$ & $99.9\%$ & $83.0\%$ & $94.2\%$ & $97.6\%$ & $99.2\%$ & $99.9\%$ \\
      \hline
      & \multicolumn{10}{c}{Intermediate electrons} \\
      \hline
      $|\tanh{s_I/10}|$  & $<0.1$ & $<0.2$ & $<0.4$ & $<0.5$ & $<0.7$ & $<0.1$ & $<0.2$ & $<0.4$ & $<0.5$ & $<0.6$ \\
      Target $\varepsilon_{bkg}$  & $10.0\%$ & $30.0\%$ & $50.0\%$ & $70.0\%$ & $90.0\%$  & $10.0\%$ & $30.0\%$ & $50.0\%$ & $70.0\%$ & $90.0\%$ \\
      Actual $\varepsilon_{bkg} $  & $8.9\%$ & $30.6\%$ & $49.8\%$ & $70.9\%$ & $90.2\%$ & $10.1\%$ & $29.3\%$ & $50.2\%$ & $69.7\%$ & $90.2\%$ \\
      Actual $\varepsilon_{sig}$  & $71.2\%$ & $92.9\%$ & $95.7\%$ & $97.1\%$ & $99.1\%$ & $70.4\%$ & $90.1\%$ & $94.4\%$ & $96.5\%$ & $98.7\%$ \\
      \hline
      & \multicolumn{10}{c}{Forward electrons} \\
      \hline
      $|\tanh{s_F/10}|$  & $<0.08$ & $<0.1$ & $<0.2$ & $<0.3$ & $<0.5$ & $<0.08$ & $<0.2$ & $<0.2$ & $<0.3$ & $<0.5$ \\
      Target $\varepsilon_{bkg}$  & $10.0\%$ & $30.0\%$ & $50.0\%$ & $70.0\%$ & $90.0\%$  & $10.0\%$ & $30.0\%$ & $50.0\%$ & $70.0\%$ & $90.0\%$ \\
      Actual $\varepsilon_{bkg} $  & $10.1\%$ & $28.9\%$ & $50.1\%$ & $70.2\%$ & $89.7\%$ & $10.7\%$ & $31.2\%$ & $49.4\%$ & $69.5\%$ & $89.9\%$ \\
      Actual $\varepsilon_{sig}$  & $57.9\%$ & $81.3\%$ & $91.0\%$ & $95.4\%$ & $98.5\%$ & $56.0\%$ & $81.1\%$ & $89.9\%$ & $95.4\%$ & $98.5\%$ \\
      \hline
    \end{tabular}
    \caption{Efficiencies for signal and background for targeted background efficiencies, as a function of $s$ for $13 \tev 25 \ns.$}
    \label{tab:eff_rej_s_beam_13_25_bkg}
  \end{center}
\end{table}
