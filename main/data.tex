\section{Data and Monte Carlo samples}

This analysis uses the following data and Monte Carlo (MC) samples:
\begin{itemize}
  \item 2012 data from the /DoubleElectron/Run2012D-ZElectron-22Jan2013-v1/RAW-RECO (\DoubleElectron) sample
  \item Multijet MC from the \texttt{/QCD\_Pt\_30\_80\_EMEnriched\_TuneZ2star\_8TeV\_pythia6/\\Summer12\_DR53X-PU25bx50\_START53\_V19D-v1/\-GEN-SIM-RAW} (\QCDLowEnergy) sample
  \item Multijet MC from the \texttt{/QCD\_Pt\_30\_80\_EMEnriched\_TuneZ2star\_13TeV-pythia6/\\Summer13dr53X-PU25bx25\_START53\_V19D-v1/\-GEN-SIM-RAW} (\QCDHighEnergy) sample
  \item $Z\to ee$ MC from the \texttt{/DYToEE\_M\_20\_TuneZ2star\_8TeV\_pythia6/\\Summer12\_DR53X-PU25bx50\_START53\_V19D-v1/\-GEN-SIM-RAW} (\ZeeLowEnergy) sample
  \item $Z\to ee$  MC from the \texttt{/DYToEE\_Tune4C\_13TeV-pythia8/\\Fall13dr-tsg\_PU20bx25\_POSTLS162\_V2-v1/GEN-SIM-RAW} (\ZeeHighEnergy) sample
  \item $t\bar{t}$ MC from the \texttt{/TT\_Tune4C\_13TeV-pythia8-tauola/\\Fall13dr-tsg\_PU20bx25\_POSTLS162\_V2-v1/GEN-SIM-RAW} (\ttbar) sample
\end{itemize}

The lumi mask \texttt{Cert\_190456-203853\_8TeV\_PromptReco\_Collisions12\_JSON.txt} is used for the \DoubleElectron\ dataset.  The QCD MC samples are enriched in jets which fake electrons, with a $p_T$ range of $30-80 \gev$.  The motivation for using the \ttbar\ sample is to ensure that the trigger paths chosen work with the latest MC campaign, as well as showing that the pixel match window behaves well in an environment with real electrons and large numbers of jets.

Events are selected which fire the trigger HLT\_Ele27\_WP80\_v13 trigger (with a second sample of events which fire the HLT\_Ele17\_Ele8\_v19 trigger as a crosscheck.)  The DoubleElectron sample was used for both triggers, as initially the study included a comparison of the $m(ee)$ spectra in data and simulation.  In addition, there is no equivalent sample /SingleElectron/Run2012D*/RAW-RECO that does not include additional requirements (eg DiTau, HighMET).

In this study, efficiencies are quoted relative to the small window, where efficiencies are estimated by simple event counting.  As a result, all windows studied are necessarily tighter than the 2012 window.
