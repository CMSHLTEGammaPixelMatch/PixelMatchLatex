\section{Variable spectra}

In addition to $\Delta\phi_1$, $\Delta\phi_2$, and $\Delta rz$, a discriminant, $s$, is defined to simplify the analysis and improve discrimination between real and fake electrons.  The definition of $s_x$ is given by:

\begin{equation}
  s_x^2 = \frac{\Delta\phi_1^2}{m(\Delta\phi_1)^2} + \frac{\Delta\phi_2^2}{m(\Delta\phi_2)^2} + \frac{\Delta rz_x^2}{m(\Delta rz_x)^2}
\end{equation}
where $x=B,I,F$, and $m(y)$ is the mean of the absolute value of the variable $y$, taken from the sum of all signal MC samples, and $rz_B=z_B$, $rz_I=r_I$, and $rz_F=r_F$.  The means of the distributions are shown in table \ref{tab:s_parameters}.

Figures \ref{fig:var_phi1_ea}-\ref{fig:var_s_ea} show the spectra for the variables per supercluster used in this analysis.  For superclusters with multiple helices, the helix with the smallest value of $s$ is chosen.  All electron helices are required to pass the small pixel window.  Due to memory constraints, only the first 50 helices are stored per supercluster.

\begin{table}[!hbt]
  \begin{center}
    \begin{tabular}{cccc}
      \hline
                        & Barrel    & Intermediate & Forward  \\
      \hline
      $m(\Delta\phi_1)$ &  $0.0069$ &     $0.0088$ &  $0.0076$ \\
      $m(\Delta\phi_2)$ & $0.00037$ &    $0.00070$ & $0.00906$ \\
      $m(\Delta z_B)$   &   $0.012$ &            - &         - \\
      $m(\Delta r_I)$   &         - &      $0.027$ &         - \\
      $m(\Delta r_F)$   &         - &            - &   $0.040$ \\
      \hline
    \end{tabular}
    \caption{The means of the variables used to calculate the $s$ variables.}
    \label{tab:s_parameters}
  \end{center}
\end{table}

\input{snippets/vars_plots_ea}
