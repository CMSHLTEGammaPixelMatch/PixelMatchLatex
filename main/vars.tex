\section{Variable spectra}

In addition to $\Delta\phi_1$, $\Delta\phi_2$, and $\Delta rz$, a discriminant, $s$, is defined to simplify the analysis and improve discrimination between real and fake electrons.  The definition of $s_x$ is given by:

\begin{equation}
  s_x^2 = \frac{\Delta\phi_1^2}{m(\Delta\phi_1)^2} + \frac{\Delta\phi_2^2}{m(\Delta\phi_2)^2} + \frac{\Delta rz_x^2}{m(\Delta rz_x)^2}
\end{equation}
where $x=B,I,F$, and $m(y)$ is the mean of the absolute value of the variable $y$, taken from the sum of all signal MC samples, and $rz_B=z_B$, $rz_I=r_I$, and $rz_F=r_F$.  The means of the distributions for signal events are shown in table \ref{tab:s_parameters}.

The spectra for the variables $\Delta\phi_1$, $\Delta\phi_2$, and $\Delta rz_2$ per supercluster are shown in figures \ref{fig:var_phi1_ea}-\ref{fig:var_rz_ea}, for signal and QCD backgrounds at $8\tev$ and $13\tev$.  The spectra for the $s$ variables are shown in figure \ref{fig:var_rz_ea}, where $\tanh{s/10}$ is plotted to stay within the range $[0,1]$.  For superclusters with multiple helices, the helix with the smallest value of $s$ is chosen.  All electron helices are required to pass the small pixel window.  Due to memory constraints, only the first 50 helices are stored per supercluster.

The spectra show high purity in the data and good agreement between the data and signal MC. In addition, there is good agreement for both the signal and background between $8\tev$ and $13\tev$, indicating similar behaviour at different energies.  The signal events tend to peak at very small values, with the backgrounds having very large tails the variables $\Delta\phi_1$, $\Delta\phi_2$, and $\Delta rz_2$, whereas for the $s$ variables the signal peaks at low values, and the backgrounds extend into a broad distribution, leading to good discrimination.

\begin{table}[!hbt]
  \begin{center}
    \begin{tabular}{cccc}
      \hline
                        & Barrel    & Intermediate & Forward  \\
      \hline
      $m(\Delta\phi_1)$ &  $0.0069$ &     $0.0088$ &  $0.0076$ \\
      $m(\Delta\phi_2)$ & $0.00037$ &    $0.00070$ & $0.00906$ \\
      $m(\Delta z_B)$   &   $0.012$ &            - &         - \\
      $m(\Delta r_I)$   &         - &      $0.027$ &         - \\
      $m(\Delta r_F)$   &         - &            - &   $0.040$ \\
      \hline
    \end{tabular}
    \caption{The means of the variables used to calculate the $s$ variables.}
    \label{tab:s_parameters}
  \end{center}
\end{table}

\begin{figure}[!bht]
  \begin{center}
    \begin{tabular}{cc}
      \includegraphics[width=0.4\textwidth]{../plots/vars/h_multivar_phi1_beam_8_50_trigger_27_B_ea} &
      \includegraphics[width=0.4\textwidth]{../plots/vars/h_multivar_phi1_beam_13_25_trigger_27_B_ea} \\
      \includegraphics[width=0.4\textwidth]{../plots/vars/h_multivar_phi1_beam_8_50_trigger_27_I_ea} &
      \includegraphics[width=0.4\textwidth]{../plots/vars/h_multivar_phi1_beam_13_25_trigger_27_I_ea} \\
      \includegraphics[width=0.4\textwidth]{../plots/vars/h_multivar_phi1_beam_8_50_trigger_27_F_ea} &
      \includegraphics[width=0.4\textwidth]{../plots/vars/h_multivar_phi1_beam_13_25_trigger_27_F_ea} \\
    \end{tabular}
  \caption{The spectra of $\Delta\phi_1$ for barrel electrons (top), intermediate electrons (middle), and forward electrons (bottom) at $8\tev$ (left) and $13\tev$ (right) for events firing the HLT\_Ele27\_WP80\_v13 trigger.}
  \label{fig:var_phi1_ea}
  \end{center}
\end{figure}
\clearpage

\begin{figure}[!bht]
  \begin{center}
    \begin{tabular}{cc}
      \includegraphics[width=0.4\textwidth]{../plots/vars/h_multivar_phi2_beam_8_50_trigger_27_B_ea} &
      \includegraphics[width=0.4\textwidth]{../plots/vars/h_multivar_phi2_beam_13_25_trigger_27_B_ea} \\
      \includegraphics[width=0.4\textwidth]{../plots/vars/h_multivar_phi2_beam_8_50_trigger_27_I_ea} &
      \includegraphics[width=0.4\textwidth]{../plots/vars/h_multivar_phi2_beam_13_25_trigger_27_I_ea} \\
      \includegraphics[width=0.4\textwidth]{../plots/vars/h_multivar_phi2_beam_8_50_trigger_27_F_ea} &
      \includegraphics[width=0.4\textwidth]{../plots/vars/h_multivar_phi2_beam_13_25_trigger_27_F_ea} \\
    \end{tabular}
  \caption{The spectra of $\Delta\phi_2$ for barrel electrons (top), intermediate electrons (middle), and forward electrons (bottom) at $8\tev$ (left) and $13\tev$ (right) for events firing the HLT\_Ele27\_WP80\_v13 trigger.}
  \label{fig:var_phi2_ea}
  \end{center}
\end{figure}
\clearpage

\begin{figure}[!bht]
  \begin{center}
    \begin{tabular}{cc}
      \includegraphics[width=0.4\textwidth]{../plots/vars/h_multivar_rz_beam_8_50_trigger_27_B_ea} &
      \includegraphics[width=0.4\textwidth]{../plots/vars/h_multivar_rz_beam_13_25_trigger_27_B_ea} \\
      \includegraphics[width=0.4\textwidth]{../plots/vars/h_multivar_rz_beam_8_50_trigger_27_I_ea} &
      \includegraphics[width=0.4\textwidth]{../plots/vars/h_multivar_rz_beam_13_25_trigger_27_I_ea} \\
      \includegraphics[width=0.4\textwidth]{../plots/vars/h_multivar_rz_beam_8_50_trigger_27_F_ea} &
      \includegraphics[width=0.4\textwidth]{../plots/vars/h_multivar_rz_beam_13_25_trigger_27_F_ea} \\
    \end{tabular}
  \caption{The spectra of $\Delta r/z$ for barrel electrons (top), intermediate electrons (middle), and forward electrons (bottom) at $8\tev$ (left) and $13\tev$ (right) for events firing the HLT\_Ele27\_WP80\_v13 trigger.}
  \label{fig:var_rz_ea}
  \end{center}
\end{figure}
\clearpage

\begin{figure}[!bht]
  \begin{center}
    \begin{tabular}{cc}
      \includegraphics[width=0.4\textwidth]{../plots/vars/h_multivar_s_beam_8_50_trigger_27_B_ea} &
      \includegraphics[width=0.4\textwidth]{../plots/vars/h_multivar_s_beam_13_25_trigger_27_B_ea} \\
      \includegraphics[width=0.4\textwidth]{../plots/vars/h_multivar_s_beam_8_50_trigger_27_I_ea} &
      \includegraphics[width=0.4\textwidth]{../plots/vars/h_multivar_s_beam_13_25_trigger_27_I_ea} \\
      \includegraphics[width=0.4\textwidth]{../plots/vars/h_multivar_s_beam_8_50_trigger_27_F_ea} &
      \includegraphics[width=0.4\textwidth]{../plots/vars/h_multivar_s_beam_13_25_trigger_27_F_ea} \\
    \end{tabular}
  \caption{The spectra of $s$ for barrel electrons (top), intermediate electrons (middle), and forward electrons (bottom) at $8\tev$ (left) and $13\tev$ (right) for events firing the HLT\_Ele27\_WP80\_v13 trigger.}
  \label{fig:var_s_ea}
  \end{center}
\end{figure}
\clearpage
